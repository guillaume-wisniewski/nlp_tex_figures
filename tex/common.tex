\usepackage{makecell}

\definecolor{alizarin}{rgb}{0.82, 0.1, 0.26}
\definecolor{airforceblue}{rgb}{0.36, 0.54, 0.66}
\definecolor{apricot}{rgb}{0.98, 0.81, 0.69}
\definecolor{blush}{rgb}{0.87, 0.36, 0.51}
\definecolor{cadmiumgreen}{rgb}{0.0, 0.42, 0.24}
\definecolor{cambridgeblue}{rgb}{0.64, 0.76, 0.68}
\definecolor{celadon}{rgb}{0.67, 0.88, 0.69}
\definecolor{chestnut}{rgb}{0.8, 0.36, 0.36}
\definecolor{harvardcrimson}{rgb}{0.79, 0.0, 0.09}
\definecolor{darkseagreen}{rgb}{0.56, 0.74, 0.56}
\definecolor{aoenglish}{rgb}{0.0, 0.5, 0.0}
\definecolor{brightube}{rgb}{0.82, 0.62, 0.91}
\definecolor{amethyst}{rgb}{0.6, 0.4, 0.8}
\definecolor{asparagus}{rgb}{0.53, 0.66, 0.42}
\definecolor{babyblue}{rgb}{0.54, 0.81, 0.94}
\definecolor{babypink}{rgb}{0.96, 0.76, 0.76}
\definecolor{amethyst}{rgb}{0.6, 0.4, 0.8}
\definecolor{bleudefrance}{rgb}{0.19, 0.55, 0.91}
\definecolor{bostonuniversityred}{rgb}{0.8, 0.0, 0.0}

\usetikzlibrary{positioning}
\usetikzlibrary{matrix}
\usetikzlibrary{fit}
\usetikzlibrary{calc,decorations.pathmorphing,patterns}
\usetikzlibrary{decorations.pathreplacing}


\newcommand{\vvector}[1]{\tikz{\draw[#1,step=1em,fill=#1!50] (0,0)  grid (1em,4em) rectangle (0, 0);}}
\newcommand{\hvector}[1]{\tikz{\draw[#1,step=1em,fill=#1!50] (0,0)  grid (4em,1em) rectangle (0, 0);}}
\newcommand{\vvectorSmall}[1]{\tikz{\draw[#1,step=.5em,fill=#1!50] (0,0)  grid (.5em,1.5em) rectangle (0, 0);}}

\newdimen\XCoord
\newdimen\YCoord
\newdimen\XXCoord
\newdimen\YYCoord
\newdimen\YaCoord
\newdimen\YbCoord

\newdimen\empty
\newdimen\fromX
\newdimen\fromY
\newdimen\toX
\newdimen\toY

\newcommand{\outgoing}[2]{
  \path (#1); \pgfgetlastxy{\XCoord}{\YCoord};
  \path (#2); \pgfgetlastxy{\XXCoord}{\YYCoord};
  \draw[->, line width=1pt] (\XCoord, \YCoord) -- (\XCoord, \YYCoord);
}%
\newcommand{\incoming}[2]{
  \path (#1); \pgfgetlastxy{\XCoord}{\YCoord};
  \path (#2); \pgfgetlastxy{\XXCoord}{\YYCoord};
  \draw[->, line width=1pt] (\XCoord, \YYCoord) -- (\XCoord, \YCoord);
}%
% horizontal arrow: (#1.x, #3.y) -- (#2.x, #3.y)
\newcommand{\horizontalArrow}[3]{
  \path (#1); \pgfgetlastxy{\fromX}{\empty};
  \path (#2); \pgfgetlastxy{\toX}{\empty};
  \path (#3); \pgfgetlastxy{\empty}{\fromY};
  \draw[->, line width=1pt] (\fromX, \fromY) -- (\toX, \fromY);
}%
% vertical braces: (#3.x, #1.y) -- (#3.x, #2.y)
\newcommand{\verticalBraces}[4]{
  \path (#1); \pgfgetlastxy{\empty}{\fromY};
  \path (#2); \pgfgetlastxy{\empty}{\toY};
  \path (#3); \pgfgetlastxy{\fromX}{\empty};
  % XXX the values for positionning the node for #4 have been choosen for the NMT example. must be checked
  \draw [decorate,decoration={brace,amplitude=3pt,raise=4pt},yshift=0pt] (\fromX, \fromY) -- (\fromX, \toY) node [midway, anchor=west, xshift=.2cm] {#4};
}%

\newcommand{\GetXCoord}[2]{%
  \pgfpointanchor{#1}{#2}%
  \pgfgetlastxy{\myx}{\dummy}%
  \myx% Renvoie la coordonnée x
}

\newcommand{\GetYCoord}[2]{%
  \pgfpointanchor{#1}{#2}%
  \pgfgetlastxy{\dummy}{\myy}%
  \myy% Renvoie la coordonnée x
}
